
\setlength{\parindent}{0pt}% Just for this example


\colorlet{itemizecolor}{black}

\begin{coloritemize}
  \item Black is Examiners Question

\end{coloritemize}

\colorlet{itemizecolor}{blue}

\begin{coloritemize}
  \item Blue is my sample question
\end{coloritemize}

\subsection{Distributed Systems - Exam paper 2015-16 Semester 7}




\begin{enumerate}
   \item Question 1 (25 marks)
   \begin{itemize}
     \item Explain, using diagrams and examples, the following terms as they apply to distributed
systems (15 marks ) : \\
Heterogeneity\\
Transparency\\
Openness\\



\begin{coloritemize}

  \item my answer
\end{coloritemize}




  \item Discuss how message queues can be used to augment traditional client-server
 communication with both synchronous and asynchronous capabilities. Illustrate your
 answer with UML diagrams where appropriate.(10 marks)

\begin{coloritemize}

  \item my answer
\end{coloritemize}





   \end{itemize}
   \item Question 2 (25 marks)
   \begin{itemize}
     \item  The following diagram depicts the system architecture for a web-based e-commerce
application that allows users to purchase concert tickets online:\\


{\includegraphics[width=0.9\textwidth]{sc1.png}}\\


While the system operates well for small to medium sized concerts, concerns have
been raised about the scalability of the system, after users experienced a major
degradation in performance when making bookings for a recent 50,000-ticket
Beyoncé concert at the Aviva Stadium.\\

You are required to provide a re-design of the system that will:\\

Allow the system to scale to support thousands of concurrent users\\


Enable the system to be extended to support the aggregation of additional
functionality from separate heterogeneous remote services.\\


Allow the system to be remotely queried using different request protocols from
different types of devices, including Windows and Linux workstations, Android
tablets and IPhones.\\




Your answer should include a fully labelled diagram of the new system architecture
along with a description of the roles of its constituent technologies and platforms(25marks)\\


\begin{coloritemize}

  \item my answer
\end{coloritemize}

   \end{itemize}


   \item Question 3 (25 marks)
   \begin{itemize}
     \item "Marshalling is a salient and essential component of distributed systems, promoting both
 loose-coupling and support for heterogeneity."\\


Discuss this statement. Your answer should address the centrality of marshalling and the role
 of middleware in distributed computing. Where appropriate, include system diagrams, UML
 designs and code snippets in your answer.(25 marks)

\begin{coloritemize}

  \item my answer
\end{coloritemize}
   \end{itemize}





      \item Question 4 (25 marks)
      \begin{itemize}
        \item Describe the function of the following components of the RMI architecture:(10 marks)\\

        The Remote Interface\\
        The RMI Registry\\
        Stubs and Skeletons\\



\begin{coloritemize}

  \item my answer
\end{coloritemize}


        \item The following UML diagram depicts the composition relationship between an Order and a
LineItem class. Both classes form part of an Order-Management system used by a local
Galway company that specialises in the sale of novelty shirts and ties.\\




{\includegraphics[width=0.9\textwidth]{sc2.png}}\\



Explain, using code examples and diagrams where appropriate, how the above classes can
be incorporated into the RMI architecture. You should apply the Open-Closed Principle in
your answer, i.e. the Order and Item classes should not require modification. You may
assume that both classes already implement the interface java.io.Serializable.(15 marks)\\


\begin{coloritemize}

  \item my answer
\end{coloritemize}
      \end{itemize}




         \item Question 5 (25 marks)
         \begin{itemize}



           \item  Discuss, citing examples, the difference between homogeneous and heterogeneous
distributed database systems.
      (8 marks)
\begin{coloritemize}

  \item my answer
\end{coloritemize}




\item Describe how the two-phase-commit protocol can be used to implement a distributed
atomic transaction.(8 marks)

\begin{coloritemize}

  \item my answer
\end{coloritemize}




           \item  Discuss how a hash-ring can be used in a distributed hash table as a mechanism for
promoting with both high availability and scalability. Include in your answer a diagram
showing how a hash ring is used to partition and locate database nodes(9 marks)

\begin{coloritemize}

  \item my answer
\end{coloritemize}
         \end{itemize}






 \item Question 6 (25 marks)
         \begin{itemize}



           \item Using a fully labelled diagram, describe the main components of a CORBA orb. Include
 in your answer a description of the services provided by the CORBA object adapter.(12 marks)

\begin{coloritemize}

  \item my answer
\end{coloritemize}




\item The following figure describes two Java interfaces that abstract a student and a class
 respectively:


{\includegraphics[width=0.9\textwidth]{sc3.png}}\\



Show how Interface Definition Language (IDL) can be used to represent these
interfaces in a CORBA architecture.


\begin{coloritemize}

  \item my answer
\end{coloritemize}



           \item   Briefly describe the mechanism through which a CORBA orb can communicate directly
 with a J2EE container.(5 Marks)

\begin{coloritemize}

  \item my answer
\end{coloritemize}
         \end{itemize}




   \end{enumerate}
