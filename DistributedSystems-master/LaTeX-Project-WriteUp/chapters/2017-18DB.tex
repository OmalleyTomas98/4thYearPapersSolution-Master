

\setlength{\parindent}{0pt}% Just for this example


\colorlet{itemizecolor}{black}

\begin{coloritemize}
  \item Black is Examiners Question

\end{coloritemize}

\colorlet{itemizecolor}{blue}

\begin{coloritemize}
  \item Blue is my sample question
\end{coloritemize}


\subsection{Distributed Systems - Exam paper 2017-18 Semester 7}





\begin{enumerate}
   \item Question 1 (25 marks)
   \begin{itemize}
     \item Briefly outline the key differences between distributed computing models and
monolithic computing models.(3 marks)


\begin{coloritemize}

  \item my answer
\end{coloritemize}


     \item Explain, using diagrams and examples, the following terms as they apply to
distributed systems:\\  Heterogeneity\\ Transparency\\ Scalability\\  Concurrency\\ (5 marks)


\begin{coloritemize}

  \item my answer
\end{coloritemize}


     \item Explain what is meant by the term Inter-Process Communication. (10 marks)\\

     Explain how each of the following Inter-Process Communication models work, and
identify the key differences between them. Use diagrams where appropriate.\\

 Remote Procedure Call Model\\

 Object-Oriented Model\\



\begin{coloritemize}

  \item my answer
\end{coloritemize}


   \end{itemize}
   \item Question 2 (25 marks)
   \begin{itemize}
     \item Explain what is meant by the term Externalisation in the context of Distributed
Systems. (10 marks)\\

Compare the following Externalisation formats, giving an example of each:\\

Binary  \\Semi-compiled \\Unicode



\begin{coloritemize}

  \item my answer
\end{coloritemize}



     \item Explain how an XML schema document may be used as a Data Definition Language
to create a platform and language neutral mechanism for data exchange. Refer to
the use of data binding, the JAXB framework and the XJC utility in your answer (10 marks).

\begin{coloritemize}

  \item my answer
\end{coloritemize}


     \item Explain using pseudocode (or Java code), how an object may be transferred from
one process to another using a Unicode format. Your answer should include the
operations performed by the client and the server.(5 marks)


\begin{coloritemize}

  \item my answer
\end{coloritemize}

   \end{itemize}


   \item Question 3 (25 marks)
   \begin{itemize}
     \item
      Describe the function of the following components of the RMI architecture, using
diagrams where appropriate:(9 marks)\\


Remote Objects and the Remote Interface\\

RMI URLs and the RMI Registry\\


Stubs and Skeletons\\


\begin{coloritemize}

  \item my answer
\end{coloritemize}


     \item Explain the procedure that is followed when creating a custom interface which
specifies how a client process may interact with a Remote object.
(10 marks)\\


You have been tasked with creating a RMI File Service. The File Service will have
the following remotely accessible methods:\\

downloadFile() – this method retrieves a file from the File Service. It takes a
file name as an argument, and will return the file requested as an array of
bytes\\

uploadFile() – this method uploads a file to the File Service. It takes the file
name and a byte array containing the file contents as arguments, and has a
void return type.\\

listFiles() – this method does not take any arguments, and returns a list of all
the files which are available to download from the File Service.\\

Write out the Java code for a Remote FileService interface which provides the
functionality described above.\\


\begin{coloritemize}

  \item my answer
\end{coloritemize}


\item Briefly describe how Java RMI can be used to provide an object façade/gateway to a
suite of server-side objects. Your answer should include a brief discussion of the
rationale for applying such an approach, and a diagram if required.(6 marks)\\


\begin{coloritemize}

  \item my answer
\end{coloritemize}

   \end{itemize}





      \item Question 4 (25 marks)
      \begin{itemize}
        \item Explain the meaning of the term SOAP, and how it is relevant to Distributed
Systems. Make reference to the SOAP Specification, Requests, Responses, and
SOAP messages in your answer. Use diagrams where appropriate.
   (9 marks)




\begin{coloritemize}

  \item my answer
\end{coloritemize}



 \item What is the function of WSDL in the context of distributed systems? List and
describe the four key aspects of a service which is described by WSDL.(8 marks)  
\begin{coloritemize}

  \item my answer
\end{coloritemize}

  \item State the principles of RESTful application development. (8 marks) 
\begin{coloritemize}

  \item my answer
\end{coloritemize}



      \end{itemize}




         \item Question 5 (25 marks)
         \begin{itemize}
           \item State Brewer’s CAP theorem, and explain the meaning of each of the three
systematic requirements to which it relates.
      (8 marks)


\begin{coloritemize}

  \item my answer
\end{coloritemize}


\item Compare the following partitioning strategies for distributed databases:(6 marks)\\

Range Partitioning\\
Hash Partitioning\\
List Partitioning\\



\begin{coloritemize}

  \item my answer
\end{coloritemize}

\item Compare the process of distributing a database across multiple nodes for the
following database types. Use diagrams where appropriate(11 marks)\\

Relational\\
Key-value store\\

\begin{coloritemize}

  \item my answer
\end{coloritemize}

         \end{itemize}


   \end{enumerate}
