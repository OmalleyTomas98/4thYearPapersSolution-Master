\setlist[coloritemize]{label=\textcolor{itemizecolor}{\textbullet}}
\colorlet{itemizecolor}{.}% Default colour for \item in itemizecolor

\setlength{\parindent}{0pt}% Just for this example


\colorlet{itemizecolor}{black}

\begin{coloritemize}
  \item Black is Examiners Question

\end{coloritemize}

\colorlet{itemizecolor}{blue}

\begin{coloritemize}
  \item Blue is my sample question
\end{coloritemize}


\subsection{Distributed Systems - Exam paper 2018-19 Semester 7}




\begin{enumerate}
   \item Question 1 (25 marks)
   \begin{itemize}
     \item Explain the importance of the following terms as they apply to distributed systems: Heterogeneity , Transparency , Scalability , Middleware , Inter-Process Communication (13 marks)



\begin{coloritemize}

  \item my answer
\end{coloritemize}






     \item Compare and contrast each of the following Inter-Process Communication models,
identifying the key differences between them. Your answer should provide an example
of each model. Use diagrams where appropriate. Remote Procedure Call Model , Object-Oriented Model , Service-Based Model 
 (12 marks)




\begin{coloritemize}

  \item my answer
\end{coloritemize}




   \end{itemize}
   \item Question 2 (25 marks)
   \begin{itemize}
     \item  “Marshalling frameworks based on highly structured Unicode formats have largely
supplanted serialisation and binary data transfer formats.”
You are required to provide a critique of this statement. Your answer should compare
Unicode and lower-level marshalling formats in terms of heterogeneity, extensibility
and efficiency. (10 marks)





\begin{coloritemize}

  \item my answer
\end{coloritemize}







     \item  “XML schema definitions in combination with data binding frameworks can greatly
simplify Inter-Process Communication in heterogeneous distributed systems”.
Provide a critique of this statement. Discuss the data modelling process, the concept
of data binding, an externalisation framework and a utility for automatically generating
the code for class definitions from a .xsd your answer.(10 marks)














\begin{coloritemize}

  \item my answer
\end{coloritemize}







     \item  Explain using pseudocode (or Java code), how an object may be transferred from one
process to another using a Unicode format. Your answer should include the operations
performed by the client and the server.(5 marks)



\begin{coloritemize}

  \item my answer
\end{coloritemize}


   \end{itemize}


   \item Question 3 (25 marks)
   \begin{itemize}
     \item Describe the function of the following components of the RMI architecture, using
diagrams where appropriate: Remote Objects  , RMI URLs and the RMI Registry
(13 marks)


\begin{coloritemize}

  \item my answer
\end{coloritemize}




     \item Explain the procedure that is followed when creating a custom interface which
specifies how a client process may interact with a Remote object. (3 marks)
Write out the Java code for a Remote interface which provides the functionality
described below. (10 marks)\\



You have been tasked with creating a RMI Database Service for student records. You
may assume that a serializable class definition Student.java is available. The methods
in the interface should make use of this serializable class definition where possible.\\
The Database Service will have the following remotely accessible methods:\\
 getStudent – this method retrieves a single student record from the database.\\
It takes an integer (student id) as an argument.
 getAllStudents – this method retrieves all student records from the database.\\
 addStudent – this method adds a new student to the database.
 deleteStudent – this method removes a single student record from the database.\\ It takes an integer (student id) as an argument.


\begin{coloritemize}

  \item my answer
\end{coloritemize}




\item Explain how a pass by reference may be simulated using the RMI framework. Use
examples of Java code and/or pseudocode to support your answer. (6 marks)



\begin{coloritemize}

  \item my answer
\end{coloritemize}

   \end{itemize}





      \item Question 4 (25 marks)
      \begin{itemize}
        \item Explain the mapping between HTTP methods and CRUD operations in RESTful
architectures.   (5 marks)


\begin{coloritemize}

  \item my answer
\end{coloritemize}
        \item Assume that a RESTful service which allows CRUD operations on a student resource
is available at the following URL: http://www.examplesite.com/students\\

Explain how a HTTP request may be made to this service to retrieve the details of a
student called Jane in XML format. Use a diagram of the client-server interaction along
with the text of a sample HTTP request and HTTP response to aid your explanation.(10 marks)








\begin{coloritemize}

  \item my answer
\end{coloritemize}



\item   Explain how annotations may be used in the JAX-RS/Jersey framework to facilitate
deployment of a Java Object as a RESTful web resource. Use a sample annotated
Java class with one method to support your answer. Your code sample should
demonstrate the use of annotations specifying the HTTP method type handled,
resource path, path parameters and MIME response type.(10 marks)

\begin{coloritemize}

  \item my answer
\end{coloritemize}
      \end{itemize}




         \item Question 5 (25 marks)
         \begin{itemize}
           \item Compare the following partitioning strategies for distributed databases: Range Partitioning , Hash Partitioning , List Partitioning
      (9 marks)



\begin{coloritemize}

  \item my answer
\end{coloritemize}


           \item State Brewer’s CAP theorem and explain the meaning of each of the three systematic
requirements to which it relates.
 (9 marks)




\begin{coloritemize}

  \item my answer
\end{coloritemize}



\item Explain the purpose of WSDL in the context of distributed systems, giving examples. List and describe the four key aspects of a service which is described by WSDL.
 (7 marks)


 \begin{coloritemize}

  \item my answer
\end{coloritemize}
         \end{itemize}






   \end{enumerate}
