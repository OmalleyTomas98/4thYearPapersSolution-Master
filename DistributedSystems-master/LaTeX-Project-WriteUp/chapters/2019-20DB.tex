\newlist{coloritemize}{itemize}{1}
\setlist[coloritemize]{label=\textcolor{itemizecolor}{\textbullet}}
\colorlet{itemizecolor}{.}% Default colour for \item in itemizecolor

\setlength{\parindent}{0pt}% Just for this example




This is a LaTeX document holding The answers/questions from 2015 exam papers to 2020 for the module 45620 -- Distributed Systems.
This .pdf acts as a study aid for any student preparing for the 2hr written paper session during Winter or Summer(Repeat).Happy reading :)
\colorlet{itemizecolor}{black}

\begin{coloritemize}
  \item Black is Examiners Question

\end{coloritemize}

\colorlet{itemizecolor}{blue}

\begin{coloritemize}
  \item Blue is my sample answer
\end{coloritemize}


\subsection{Distributed Systems - Exam paper 2019-20 Semester 7}




\begin{enumerate}
   \item Question 1 (25 marks)
   \begin{itemize}
     \item  Explain what is meant by heterogeneity in distributed systems, giving examples of
ways in which distributed systems can be heterogeneous. (15) marks

\begin{coloritemize}

\item heterogeneity is one of the two key attributes of Distributed Systems (1.Resource Sharing 2.heterogeneity)
heterogeneity in simple terms describes the variety and difference in systems. Describes the Variety and difference in modern
computing obstacles such as Networks , Hardware and Operating Systems.
\end{coloritemize}








     \item What is the role of middleware in a distributed system? (5 marks)


\begin{coloritemize}

\item  The role of middleware is to make application development easier, by providing common
programming abstractions, by masking the heterogeneity and the distribution of the
underlying hardware and operating systems, and by hiding low-level programming details


\end{coloritemize}



     \item Explain what is meant by distribution transparency in a distributed system, giving
examples of types of transparency. (5 marks)





\begin{coloritemize}

 \item In a distributed systems there are some features of the system which are hidden from the
 users this is called transparency. The end users are not aware of certain mechanisms
  which do not appear on the distributed applications
because transparency confines it in the layer below the one user interacts with.
There are several types of transparencies which are available to be implemented on
distributed systems such as access transparency, location transparency, concurrency transparency,
 replication transparency, failure transparency, scaling transparency etc.

\end{coloritemize}


   \end{itemize}
   \item Question 2 (25 marks)
   \begin{itemize}
     \item  Explain what is meant by Data Serialisation, and discuss why it is necessary for
distributed systems. (5 marks)



\begin{coloritemize}

  \item my answer
\end{coloritemize}


     \item Discuss the relative merits of the following types of Serialisation formats, and give a
specific example of each type: – Text-Based – Semi-Compiled – Binary (15 marks).


\begin{coloritemize}

  \item my answer
\end{coloritemize}

     \item  Write a Protocol Buffer message definition for a Person message, with fields name
(string), id (integer), and email (string). Either proto2 or proto3 syntax is acceptable. (5 marks)



\begin{coloritemize}

  \item my answer
\end{coloritemize}
   \end{itemize}


   \item Question 3 (25 marks)
   \begin{itemize}
     \item Processes can communicate with each other by passing messages in different ways.
Explain what is meant by the following types of inter-process messaging: – Persistent – Transient  – Synchronous – Asynchronous
(10 marks)



\begin{coloritemize}

  \item my answer
\end{coloritemize}


     \item Describe the functioning of the Remote Procedure Call (RPC) model for inter-process
communication, using diagrams as necessary, and the gRPC framework as a specific
example. (15 marks)



\begin{coloritemize}

  \item my answer
\end{coloritemize}
   \end{itemize}





      \item Question 4 (25 marks)
      \begin{itemize}
        \item Discuss the ways in which the REST architectural style for web services differs from
traditional web services based on SOAP and XML-RPC. Your answer should include
a description of REST’s reliance on web protocols.
   (15 marks)


   \begin{coloritemize}

  \item my answer
\end{coloritemize}
        \item Explain what the OpenAPI specification is, and discuss the role of OpenAPI/Swaggger in developing RESTful web services. (10 marks)

        \begin{coloritemize}

  \item my answer
\end{coloritemize}
      \end{itemize}




         \item Question 5 (25 marks)
         \begin{itemize}
           \item Replication and Partitioning are fundamental techniques employed in the design and
implementation of distributed data stores. Explain what is meant by these two terms,
and discuss why they are useful.
      (10 marks)


      \begin{coloritemize}

  \item my answer
\end{coloritemize}


           \item Describe the MapReduce programming model for distributed batch processing of
large datasets. (10 marks)


\begin{coloritemize}

  \item my answer
\end{coloritemize}
\item Give an example of how MapReduce could be used to determine the frequency with
which different URLs are accessed based on logs of web page requests. (5 marks)


\begin{coloritemize}

  \item my answer
\end{coloritemize}
         \end{itemize}






   \end{enumerate}
