
\setlength{\parindent}{0pt}% Just for this example


\colorlet{itemizecolor}{black}

\begin{coloritemize}
  \item Black is Examiners Question

\end{coloritemize}

\colorlet{itemizecolor}{blue}

\begin{coloritemize}
  \item Blue is my sample question
\end{coloritemize}

\subsection{Artificial Intelligence - Exam paper 2015-16 Semester 8}




\begin{enumerate}
   \item Question 1 (25 marks)
   \begin{itemize}
     \item“Brute force recursive search algorithms are the most versatile and adaptable
mechanisms for traversing a semantic network or search tree.”\\
Provide a critique of this statement that addresses both the space and time complexity
of the depth-first approach and show how the weaknesses of the algorithm can be
overcome through depth limitation and iterative deepening. Use diagrams and code
snippets to illustrate your answer.
(25 Marks)


\begin{coloritemize}

  \item my answer
\end{coloritemize}






   \end{itemize}
   \item Question 2 (25 marks)
   \begin{itemize}
     \item  Figure 1 depicts a semantic network of nodes interconnected by edges. The
starting node is node ‘A” and “I” the goal node. Each node is labelled with a letter and
a heuristic estimate of distance to the goal node. The actual distance between two
nodes is shown as a number along their connecting edge.\\



\begin{center}




{\includegraphics[width=0.5\textwidth]{t1.png}}\\

\end{center}







\item (a) Show how the A* algorithm can find the optimal path from the initial node (A) to the
goal node (I). Your answer should show the state of the OPEN and CLOSED queues
for each iteration of the algorithm and also show how the path evaluation function,
f(n), is computed.
(15 marks)

\begin{coloritemize}

  \item my answer
\end{coloritemize}



\item (b) Using either pseudocode or Java to illustrate your answer, provide a critique of those
parts of the A* algorithm that contribute to its optimality and completeness.
(10 Marks)

\begin{coloritemize}

  \item my answer
\end{coloritemize}

   \end{itemize}


   \item Question 3 (25 marks)
   \begin{itemize}
     \item Figure 2 below depicts a 4-ply game tree, with leaf nodes labelled with a
score that represents a goal state.\\




\begin{center}




{\includegraphics[width=0.5\textwidth]{t2.png}}\\

\end{center}










\item  (a) Show, using labelled diagrams, how the minimax algorithm can determine
the best move to make from node ‘A’. Your answer should also illustrate how MAX
and MIN values are computed at each level.
(10 Marks)
\begin{coloritemize}

  \item my answer
\end{coloritemize}






\item (b) Describe how alpha-beta pruning can be applied to the game tree in Figure 2 to
reduce the number of nodes to be generated and examined. Your answer should show
the pruned game tree, indicate the alpha and beta cut-off points and address the
effectiveness of alpha-beta pruning.
(15 Marks)
\begin{coloritemize}

  \item my answer
\end{coloritemize}
   \end{itemize}





      \item Question 4 (25 marks)
      \begin{itemize}
        \item . (a) “Branching factor is the most salient characteristic that determines the effectiveness
of a search algorithm on a semantic network.”\\
Discuss this statement and evaluate implications of branching factor for both the
space and time complexity of a search strategy.
(13 Marks)



\begin{coloritemize}

  \item my answer
\end{coloritemize}


        \item) Discuss the application of breadth-first search and its heuristically informed variants
to semantic trees and networks. Your discussion should address the impact of each
approach on search optimality, completeness and space complexity. Include
diagrams and algorithms, in either pseudocode or Java, with your answer.
(12 Marks)


\begin{coloritemize}

  \item my answer
\end{coloritemize}
      \end{itemize}




         \item Question 5 (25 marks)
         \begin{itemize}



           \item  . (a) Explain the following terms as they apply to fuzzy logic:\\
- Membership Functions (5 Marks)\\
- Hedges (5 Marks)\\
- Fuzzy Set Operations (5 Marks)\\
\begin{coloritemize}

  \item my answer
\end{coloritemize}




\item (b) Discuss how fuzzy rules are evaluated and aggregated in the Mamdani inference
model. Your answer should include sample rules and diagrams where appropriate.
(10 Marks)


\begin{coloritemize}

  \item my answer
\end{coloritemize}

         \end{itemize}


 \item Question 6 (25 marks)
         \begin{itemize}



           \item (a) Explain the following terms as they apply to artificial neural networks (ANNs):\\
- Activation Functions (3 Marks)\\
- Weight Training (3 Marks)\\


\begin{coloritemize}

  \item my answer
\end{coloritemize}




\item (b) Using a fully labelled diagram, describe the structure of a perceptron and show how a
perceptron can learn classification tasks.
(10 Marks)


\begin{coloritemize}

  \item my answer
\end{coloritemize}

           \item   (c) Describe the structure and function of a multilayer back-propagation neural network.
Your answer should include a diagram that illustrates the direction of information flow
through the network.
(9 Marks)

\begin{coloritemize}

  \item my answer
\end{coloritemize}
         \end{itemize}




   \end{enumerate}
