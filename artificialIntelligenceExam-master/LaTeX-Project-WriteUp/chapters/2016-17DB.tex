
\setlist[coloritemize]{label=\textcolor{itemizecolor}{\textbullet}}
\colorlet{itemizecolor}{.}% Default colour for \item in itemizecolor

\setlength{\parindent}{0pt}% Just for this example


\colorlet{itemizecolor}{black}

\begin{coloritemize}
  \item Black is Examiners Question

\end{coloritemize}

\colorlet{itemizecolor}{blue}

\begin{coloritemize}
  \item Blue is my sample question
\end{coloritemize}

\subsection{Artificial Intelligence - Exam paper 2016-17 Semester 8}


\begin{enumerate}
   \item Question 1 (25 marks)
   \begin{itemize}
     \item  1. (a) Using examples where appropriate, describe each of the following terms as they apply to
artificial neural networks:\\
- Activation Functions (5 Marks)\\
- Perceptrons (5 Marks)\\

\begin{coloritemize}

  \item my answer
\end{coloritemize}





     \item (b) Discuss how backpropagation can be used to train a multi-layered neural network. Your
answer should include a diagram and explain each step in the process.
(15 Marks)\\


\begin{coloritemize}

  \item my answer
\end{coloritemize}


   \end{itemize}
   \item Question 2 (25 marks)
   \begin{itemize}


     \item Figure 1 below depicts a semantic network of nodes interconnected by edges. The
starting node is node ‘A” and “I” is the goal node. Each node is labelled with a letter
in the upper compartment and a heuristic estimate of distance to the goal node in the
lower compartment. The actual distance between two nodes is shown as a number
along their connecting edge\\





\begin{center}



{\includegraphics[width=0.5\textwidth]{a1.png}}\\

\end{center}


\item (a) Show how the A* algorithm can find the optimal path from the initial node (A) to the
goal node (I). Your answer should show the state of the OPEN and CLOSED queues
for each iteration of the algorithm and how the path evaluation function, f(n), is
computed.
(13 Marks)

\begin{coloritemize}

  \item my answer
\end{coloritemize}

\item (b) Discuss the factors that impact on the space complexity of the A* algorithm and
explain, using examples, how iterative deepening can reduce the memory overhead
whilst preserving both optimality and completeness.
(12 Marks)


\begin{coloritemize}

  \item my answer
\end{coloritemize}  




 \end{itemize}


   \item Question 3 (25 marks)
   \begin{itemize}




     \item  (a) Discuss each of the following topics as they relate to semantic networks:\\
• Branching Factor (6 Marks)\\
• Brute Force Search (6 Marks)\\


\begin{coloritemize}

  \item my answer
\end{coloritemize}


\item (b) Describe the structure and function of a Radial Basis Function (RBF) neural
network. Your answer should include a diagram explaining the role of each layer in
the network topology.
(13 Marks)


\begin{coloritemize}

  \item my answer
\end{coloritemize}   \end{itemize}



      \item Question 4 (25 marks)
      \begin{itemize}





        \item Figure 2 below depicts a 4-ply full game tree having leaf nodes decorated
with a score that represents a goal state. You may assume that node ‘A’ is a MAX
node.



\begin{center}




{\includegraphics[width=0.5\textwidth]{a2.png}}\\

\end{center}




   \item (a) Show, using labelled diagrams, how the minimax algorithm can determine the best
move to make from node ‘A’. Your answer should also illustrate how MAX and MIN
values are computed at each level.
(10 Marks)


  

\begin{coloritemize}

  \item my answer
\end{coloritemize}
  \item  (b) Describe how alpha-beta pruning can be applied to the game tree in Figure 2 to
reduce the number of nodes to be generated and examined. Your answer should show
the pruned game tree, indicate the alpha and beta cut-off points and address the
effectiveness of alpha-beta pruning.
(15 Marks)


\begin{coloritemize}

  \item my answer
\end{coloritemize}

      \end{itemize}




         \item Question 5 (25 marks)
         \begin{itemize}





           \item (a) Explain the fuzzy set operations that underpin fuzzy logic and explain how they relate
to Boolean logic. Include diagrams with your answer where appropriate.
(10 Marks)


\begin{coloritemize}

  \item my answer
\end{coloritemize}
\item (b) Figures 3, 4 and 5 below depict fuzzy sets that describe the variables growth, taxtake and spending respectively. The universe of discourse ranges from 0 - 10 percent for
the variables growth and tax-take and from 0-25 percent for the variable spending. The
following three rules describe the reasoning used by a fuzzy inference system to
compute the increase in government spending for the inputs growth and taxtake:\\


1. if growth is very strong and taxtake is not small then spending is high\\
2. if growth is weak or taxtake is not normal then spending is low\\
3. if growth is moderate and taxtake is somewhat normal then spending is
balanced\\
Compute, using the Sugeno inference method, the crisp value of spending for the
inputs growth=7.5 and taxtake=3. Your answer should clearly show each step in the
fuzzy inference process and how the output values are computed.
(15 Marks)




\begin{center}




{\includegraphics[width=0.5\textwidth]{a3.png}}\\

\end{center}





\begin{center}




{\includegraphics[width=0.5\textwidth]{a4.png}}\\

\end{center}





\begin{center}




{\includegraphics[width=0.5\textwidth]{a5.png}}\\

\end{center}




\begin{coloritemize}

  \item my answer
\end{coloritemize}



         \end{itemize}






   \end{enumerate}
