

\setlength{\parindent}{0pt}% Just for this example


\colorlet{itemizecolor}{black}

\begin{coloritemize}
  \item Black is Examiners Question

\end{coloritemize}

\colorlet{itemizecolor}{blue}

\begin{coloritemize}
  \item Blue is my sample question
\end{coloritemize}


\subsection{Artificial Intelligence - Exam paper 2017-18 Semester 8}





\begin{enumerate}
   \item Question 1 (25 marks)
   \begin{itemize}
     \item  (a) Describe, using examples where appropriate, how an Artificial Neural Network
(ANN) can be trained to learn classification tasks. Include a fully labelled diagram in
your answer, showing the structure of both a neuron and a perceptron.
(15 Marks)


\begin{coloritemize}

  \item my answer
\end{coloritemize}


     \item Discuss the structure and function of a multilayer back-propagation neural network.
Your answer should include a diagram that illustrates the direction of information
flow through the network and address techniques for choosing the network topology.
(10 Marks)


\begin{coloritemize}

  \item my answer
\end{coloritemize}



\begin{coloritemize}

  \item my answer
\end{coloritemize}


   \end{itemize}
   \item Question 2 (25 marks)
   \begin{itemize}
     \item Figure 1 below depicts a 4-ply game tree having leaf nodes decorated
with a score that represents the computation of a static evaluation function:


\begin{center}




{\includegraphics[width=0.5\textwidth]{f1.png}}\\

\end{center}



     \item (a) Show, using labelled diagrams, how the minimax algorithm can determine the best
move to make from node ‘A’. Your answer should clearly illustrate how MAX and
MIN values are computed at each level.
(10 Marks)

\begin{coloritemize}

  \item my answer
\end{coloritemize}


     \item (b) Describe how alpha-beta pruning can be applied to the game tree in Figure 1 to
reduce the number of nodes to be generated and examined. Your answer should show
the pruned game tree, indicate the alpha and beta cut-off points and address the
computational effectiveness of alpha-beta pruning.
(15 Marks)

\begin{coloritemize}

  \item my answer
\end{coloritemize}

   \end{itemize}


   \item Question 3 (25 marks)
   \begin{itemize}
     \item
   (a) “Branching factor is the one characteristic of an algorithm, more than any other, that
will determine the effectiveness of a search strategy on a semantic network.”\\


Discuss this statement and evaluate implications of branching factor for the
computational efficiency of a search strategy.
(12 Marks)\\


\begin{coloritemize}

  \item my answer
\end{coloritemize}


     \item Discuss the limitations of the basic hill-climbing algorithm and how they may be
mitigated by steepest-ascent and simulated annealing techniques. Use diagrams and
pseudocode or Java snippets to illustrate your answer.
(13 Marks)\\


\begin{coloritemize}

  \item my answer
\end{coloritemize}


   \end{itemize}





      \item Question 4 (25 marks)
      \begin{itemize}
        \item EFigure 2 below depicts a semantic network of ten nodes interconnected by edges. The
starting node is node ‘A” and “J” is the goal node. Each node is labelled with a letter
in the upper compartment and a heuristic estimate of distance to the goal node in the
lower compartment. The actual distance between two nodes is shown as a number
along their connecting edge.


\begin{center}




{\includegraphics[width=0.5\textwidth]{f2.png}}\\

\end{center}



 \item(a) Show how the A* algorithm can find the optimal path from the initial node (A) to the
goal node (J). Your answer should clearly show the state of the OPEN and CLOSED
queues for each iteration of the algorithm and how the path evaluation function, f(n),
is computed.
(11 Marks)
\begin{coloritemize}

  \item my answer
\end{coloritemize}

  \item (b) Discuss the efficiency of the A* algorithm and the parts of the algorithm that
contribute most to the computational complexity of the search of a semantic network.
Your answer should also address how different types of graph topologies may impact
the performance of A*.(6 marks)
\begin{coloritemize}

  \item my answer
\end{coloritemize}


\item (c) Using a diagram and code snippets where appropriate, discuss how iterative
deepening can be applied to A* to reduce computational complexity without
compromising algorithmic optimality and completeness.
(8 Marks)
\begin{coloritemize}

  \item my answer
\end{coloritemize}

      \end{itemize}


         \item Question 5 (25 marks)
         \begin{itemize}
           \item. (a) Explain the following terms as they apply to fuzzy logic:\\
- Membership Functions (4 Marks)\\
- Hedges (4 Marks)\\


\begin{coloritemize}

  \item my answer
\end{coloritemize}


\item Compare the following partitioning strategies for distributed databases:(6 marks)\\

Range Partitioning\\
Hash Partitioning\\
List Partitioning\\



\begin{coloritemize}

  \item my answer
\end{coloritemize}

\item(b) A college has created a result forecasting system based on fuzzy logic that computes a
percentage based on input values of course difficulty and CAO points. Figures 3, 4
and 5 below depict fuzzy sets that describe the linguistic variables course, points and
result respectively that are used by the forecasting system. The universe of discourse
ranges from 0 – 10 for the variable course and from 200 – 600 for the variable points.
The linguistic variable result has a universe of discourse spanning the range 0-100 percent\\


The following three rules describe the reasoning used by a fuzzy inference system for
computing a percentage result for the inputs course and points:\\


- If course is difficult and points is not high then result is poor\\
- If course is easy then result is good\\
- If course is normal and points is average then result is mediocre\\

Compute, using the Mamdani inference method and a Right-Most-Max defuzzifier,
the predicted result for a student with 450 points that has taken a course rated with a
level of difficulty of 6.5. Your answer should clearly show each step in the fuzzy
inference process.
(17 Marks)




\begin{center}




{\includegraphics[width=0.5\textwidth]{f3.png}}\\

\end{center}




\begin{center}




{\includegraphics[width=0.5\textwidth]{f4.png}}\\

\end{center}




\begin{center}




{\includegraphics[width=0.5\textwidth]{f4.png}}\\

\end{center}



\begin{coloritemize}

  \item my answer
\end{coloritemize}

         \end{itemize}


   \end{enumerate}
