
\newlist{coloritemize}{itemize}{1}

\setlist[coloritemize]{label=\textcolor{itemizecolor}{\textbullet}}
\colorlet{itemizecolor}{.}% Default colour for \item in itemizecolor

\setlength{\parindent}{0pt}% Just for this example




This is a LaTeX document holding The answers/questions from 2015 exam papers to 2019 for the module COMP08016 -- Artificial Intelligence.
This .pdf acts as a study aid for any student preparing for the 2hr written paper session during May or August(Repeat).Best of Luck


\colorlet{itemizecolor}{black}

\begin{coloritemize}
  \item Black is Examiners Question

\end{coloritemize}

\colorlet{itemizecolor}{blue}

\begin{coloritemize}
  \item Blue is my sample question
\end{coloritemize}


\subsection{Artificial Intelligence - Exam paper 2018-19 Semester 8}


\begin{enumerate}
   \item Question 1 (25 marks)
   \begin{itemize}
     \item Describe, using examples where appropriate, how an Artificial Neural Network
          (ANN) can be trained to learn classification tasks that are linearly-separable. Include
          a fully labelled diagram in your answer, showing the structure of a perceptron and
          explain the key steps involved in the training process.
          (15 Marks)



\begin{coloritemize}

  \item my answer
\end{coloritemize}






     \item Discuss the structure and function of a multilayer back-propagation neural network.
            Your answer should include a diagram that illustrates the direction of information
            flow through the network, how errors are back-propagated and address the techniques
            for choosing a network topology.
            (10 Marks)


\begin{coloritemize}

  \item my answer
\end{coloritemize}




   \end{itemize}
   \item Question 2 (25 marks)
   \begin{itemize}
     \item Figure 1 below depicts a semantic network of eight nodes interconnected by edges.
The starting node is node ‘A” and “H” is the goal node. Each node is labelled with a
letter in the upper compartment and a heuristic estimate of distance to the goal node in
the lower compartment. The actual distance between two nodes is shown as a number
along their connecting edge.


\begin{center}




{\includegraphics[width=0.5\textwidth]{q1.png}}\\

\end{center}






\begin{coloritemize}

  \item my answer
\end{coloritemize}







     \item  (a) Show how the A* algorithm can find the optimal path from the initial node (A) to the
goal node (H). Your answer should clearly show the state of the OPEN and CLOSED
queues for each iteration of the algorithm and how the path evaluation function, f(n),
is computed.
(11 Marks)




\begin{coloritemize}

  \item my answer
\end{coloritemize}







     \item (b) Discuss the efficiency of the A* algorithm and the parts of the algorithm that
contribute most to the computational complexity of the search of a semantic network.
Your answer should also address how different types of graph topologies may impact
the performance of A*.
(6 Marks)


\begin{coloritemize}

  \item my answer
\end{coloritemize}











 \item (c) Using a diagram and code snippets where appropriate, discuss how iterative
deepening can be applied to A* to reduce computational complexity without
compromising algorithmic optimality and completeness.
(8 Marks)


\begin{coloritemize}

  \item my answer
\end{coloritemize}


   \end{itemize}


   \item Question 3 (25 marks)
   \begin{itemize}
     \item Figure 2 below depicts a 4-ply game tree having leaf nodes decorated
with a score that represents the computation of a static evaluation function:

\begin{center}




{\includegraphics[width=0.5\textwidth]{q1.png}}\\

\end{center}

\begin{coloritemize}

  \item my answer
\end{coloritemize}




     \item (a) Show, using labelled diagrams, how the minimax algorithm can determine the best
move to make from node ‘A’. Your answer should clearly illustrate how MAX and
MIN values are computed at each level.
(10 Marks)



\begin{coloritemize}

  \item my answer
\end{coloritemize}




\item  (b) Describe how alpha-beta pruning can be applied to the game tree in Figure 2 to
reduce the number of nodes to be generated and examined. Your answer should show
the pruned game tree, indicate the alpha and beta cut-off points and address the
computational effectiveness of alpha-beta pruning.
(15 Marks)



\begin{coloritemize}

  \item my answer
\end{coloritemize}

   \end{itemize}





      \item Question 4 (25 marks)
      \begin{itemize}
        \item (a) Explain, using examples, the following terms as they apply to heuristic search
algorithms:\\
- Admissibility \\
- Monotonicity \\
- Foothills and Plateaux\\

\begin{coloritemize}

  \item my answer
\end{coloritemize}
        \item (b) Discuss how steepest-ascent can overcome the limitations of the basic hill-climbing
algorithm and contrast hill-climbing and best-first approaches. Use diagrams and
pseudocode or Java snippets to illustrate your answer.
(10 Marks)





\begin{coloritemize}

  \item my answer
\end{coloritemize}


\item (c) Explain how simulated annealing can be used to overcome the limitations of hill-climbing
algorithms when searching for an optimal solution to a search problem.
(6 Marks)
\begin{coloritemize}

  \item my answer
\end{coloritemize}
      \end{itemize}




         \item Question 5 (25 marks)
         \begin{itemize}
           \item (a) Explain the fuzzy set operations that underpin fuzzy logic and explain how they
relate to Boolean logic. Include diagrams with your answer where appropriate.(8 marks)



\begin{coloritemize}

  \item my answer
\end{coloritemize}


           \item  (b) An autonomous car has a braking system implemented using fuzzy logic, that
applies pressure to a brake in response to the distance of an object ahead of it and the
amount of rain on a road surface. Figures 3, 4 and 5 below depict fuzzy sets that
describe the linguistic variables distance, road and brake respectively that are used
by the braking system. The universe of discourse ranges from 0 – 100 metres for the
variable distance and from 0 – 5mm for the variable road. The linguistic variable
brake is defined using singleton spikes and has a universe of discourse spanning the
range 0-100 percent \\

The following three rules describe the reasoning used by a fuzzy inference system:\\

-  IF distance IS very very short OR road IS extremely dry THEN brake IS more
or less hard \\
- IF distance IS not slightly long AND road IS somewhat moist THEN brake IS
not soft \\
- IF distance IS medium AND road IS not very wet THEN brake IS very normal \\

Compute, using the Sugeno inference method, the predicted braking pressure with
input parameters of distance = 30 metres and road = 3mm of rain. Your answer
should clearly show each step in the fuzzy inference process.
(17 Marks)


\begin{center}




{\includegraphics[width=0.5\textwidth]{q3.png}}\\

\end{center}


\begin{center}




{\includegraphics[width=0.5\textwidth]{q4.png}}\\

\end{center}





\begin{center}




{\includegraphics[width=0.5\textwidth]{q5.png}}\\

\end{center}



\begin{coloritemize}

  \item my answer
\end{coloritemize}






\begin{center}


Formula

{\includegraphics[width=0.5\textwidth]{table.png}}\\

\end{center}

 
         \end{itemize}






   \end{enumerate}
